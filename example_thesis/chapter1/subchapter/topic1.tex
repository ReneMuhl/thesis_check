In der indirekten Rede verwendet man in der Regel die Form des Konjunktivs I (coniunctivus obliquus). Wenn die Formen des Indikativs und des Konjunktivs I gleich sind, wird auf die Formen des Konjunktivs II zurückgegriffen, um die Mittelbarkeit des Gesagten zu verdeutlichen. Sind auch die entsprechenden Konjunktiv-II-Formen identisch mit Indikativformen, so kann die entsprechende Konjunktiv-II-Form mit „würde“ benutzt werden. Eine Formengleichheit zwischen Indikativ- und Konjunktivformen besteht immer in der 1. und 3. Person Plural (wir/sie) und meist (bei regelmäßigen Verben immer) in der 1. Person Singular (ich). Am häufigsten werden in indirekter Rede Aussagen in der 3. Person wiedergegeben.
Zum Ausdruck der Vorzeitigkeit des Geschehens wird die Vergangenheitsform (Perfekt-Form) des Konjunktivs I, bei Ausdruck der Gleichzeitigkeit die Gegenwartsform (Präsens-Form) des Konjunktivs I, zur Darstellung einer Nachzeitigkeit die Zukunftsform (Futur-Form) des Konjunktivs I verwendet. Der Konjunktiv I nimmt im Deutschen nicht die Tempusform an, die der Hauptsatz aufweist. Als Bezugspunkt für die Beurteilung der Nach-, Gleich- und Vorzeitigkeit ist der Zeitpunkt der Äußerung durch den Dritten maßgeblich. Es ist nicht auf den Zeitpunkt der indirekten Wiedergabe abzustellen.
Für den Konjunktiv I stehen daher nur drei Zeitformen zur Verfügung, die im Folgenden an Beispielen erklärt werden.